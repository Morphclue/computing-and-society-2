\section{Participatory Design}
Participatory design is another design method that helps to further increase the acceptance rate of UK4You\cite{participatory-design-spinuzzi}.
It is also often referred to as co-design.
Participatory design consists of going through the design process together with end users or other stakeholders.
Thus, the wishes of the users are integrated into the platform similarly to the user-centered design methodology.
End users can also draw designers' attention to things that might otherwise not have been noticed.
However, it is important to mention here that with co-design, the main designer remains the decision-maker.\\

In order to implement this design method, two students were asked to become co-designers.
They were shown all the knowledge they had already acquired.
For the final platform design, various snippets were put together for a paper prototype.
Together with the two other students, the individual snippets were rearranged, redesigned or discarded.
After each small iteration, the platform could be improved.
It was always important that the values of productivity and inclusiveness defined at the beginning were observed.
In particular, it was not developed for users, but for the two personas Stefan and Maria.\\

As part of the design phase, three elementary website pages were created, which can be found under \autoref{sec:mockups}.
\autoref{fig:timeline} shows the homepage with the timeline as the central element.
\autoref{fig:company} provides the opportunity to look at the different companies.
Another core feature is the profile page, which is shown in \autoref{fig:profile}.
The designs created are high fidelity mockups created from the previously mentioned paper wireframes which have already gone through several iteration stages.\\

First, a central element is described, which can be found in all three mockups.
The navigation bar, which contains the logo, a search bar, and three other buttons.
The first thing you notice is that the navigation bar is designed in the university colors\cite{uni-kassel-colors}.
These are used to further support the university's corporate design.\\

The choices for the logo or the title are not further explained here.
The search bar, on the other hand, needs some explanation because search results need to be filtered on a specific basis.
It is particularly important here that users have the power to decide for themselves how they want their content to be displayed.
For this reason, it should be possible to change the sorting of the search results, the timeline or the company page, similar to Reddit.
Sorting could be ascending or descending based on interactions from other users.
However, other sorting options could also be used, such as relevance to specific users.
For example, in the case of company offers, Stefan could receive job offers with a lot of practical experience and Maria with a lot of scientific progress.
Depending on your preferences, it should also be possible to filter the content according to specific criteria.
So Maria could reduce her posts on the timeline to academic and conferences and Stefan could filter out exactly this content.\\

In addition, there are three different buttons, which indicate some functionalities as an example.
The bell should represent the notification system.
Mentions, answers or very important information for users should be displayed there.
The notification system is also customizable down to the smallest detail.
Finally, the importance of a notification can be weighted differently by students.
It is also important that, except urgent messages from the university, all notifications are opt-in instead of opt-out.
Therefore, users are not spammed at the beginning without having given their consent.\\

The paper airplane symbol represents the message system.
Here people on the UK4You platform can get in touch.
The messaging system is similar to the other platforms mentioned here, such as Facebook, WhatsApp or Twitter.
By default, a time is sent, and it is also displayed whether users have read the message.
In contrast to the other platforms, however, these features are also opt-in instead of opt-out.
In this way, it cannot happen that undesired information is disclosed at the beginning that should not be disclosed.\\

The last icon found in the navigation bar is the person icon.
If you click on this, a dropdown menu should appear, which leads to the profile on the one hand, but also has setting options.
Here it should be possible to get to a configuration page on which you can change typical elements such as password or e-mail.
At this point it should be noted that this is currently the only way to navigate to other pages, along with another option.
The other option is to click on the navigation bar title to return to the timeline view.
Currently, no more navigation options have been created because the focus was initially on the general design.
However, there is already the idea of inserting navigation elements on the left side, similar to Facebook.\\

Now that the navigation bar has been explained in detail, the timeline in \autoref{fig:timeline} is explained in more detail.
You can first see that a lot of space has been left on both the left and the right side of the timeline.
As mentioned before, many students primarily use their mobile devices to access social media.
For this reason, consideration was given in advance to minimizing the effort for developers and offering a unified view for a pc and mobile view.\\

The timeline is a listing of posts, the listing of which can be optimized by users themselves.
In this example, Maria is shown a view based on her preference for academic content.
There are different post types in this mockup.
In this example there are news, a conference paper and course updates on theoretical computer science 2.
There are many other types of posts besides these examples.
In addition to the type of post, there is also the author.
The content of the post has a gray background in the \autoref{fig:timeline}.\\

There are different ways to interact with this content using three different buttons.
Posts can be liked by users.
However, these likes cannot be viewed by other people and are only used to promote a specific filter function.
This way, users are not discouraged from posting content or have to compare themselves to content with more likes.
There is also an easy way to share content with other users directly via the messaging system.
In this way, content can be easily exchanged between users.\\

Between these two buttons there is the comment function.
Just like posts, comments can be problematic for a variety of reasons.
Many of these have already been explained in the previous sections.
Based on this acquired knowledge, it becomes clear that without a filter or the like, one can be offended.
A logical consequence of this is that content is initially set on hold until a moderator has checked it.\\

However, this possibility is not completely error-free.
This can be shown with an example list\cite{profanity-list}.
A profanity word filter could be used to recognize words that are not meant as an insult.
The word Willy could also be a normal name or a pet name for William.
Also, while many variations of the profane word shit will be caught (s\_h\_i\_t, sh!+, sh!t, sh1t, etc.), it will also be possible to bypass the blacklist with new variations (\$hit).
So it becomes clear that there will be profanity words that are wrongly recognized as well as profanity words that are not discovered at all.
For this reason, on the one hand, it should be possible to unblock incorrectly recognized insults or words and, on the other hand, to send insulting content to the moderation team.\\

At this point, the question arises as to what should happen to such users.
Users who intentionally offend or act destructively and do so repeatedly should be given restricted access to UK4You.
They should not be completely ruled out because the platform continues to provide important information.
For this reason, only read rights are released for the person.
After a certain time and after understanding the person, a moderator can talk about lifting this restriction.\\

The three dots on a post offer further possibilities to interact with this post.
Here the post can also be flagged as problematic and sent to the moderation team.
It should also be possible to hide the post or avoid further posts of this type.\\

Since transparency is very important at UK4You, similar to ResearchGate, it is also listed why a post was suggested.
Although ResearchGate lists that the post is suggested, for example, based on the selected interests, it does not specify exactly which interest it is.
This is of course problematic because suggestions are harder to understand. 
For this reason, UK4You provides as much detailed information as possible.
This information is also highlighted in color to increase the user's awareness of their content.\\

Furthermore, the awareness should also be strengthened where users are unsure whether they have already seen certain content.
Similar to Instagram, a dividing line with already seen should draw attention to the fact that content has already been seen.
This also counteracts the infinite scroll and in turn increases the productive time on the platform.\\

% TODO: suggestions from PD
% TODO: Course to use the software

% Later CS2 presentation:
% TODO: no friends but messaging? -> first idea to avoid showing friend group etc, but too stuck on these thoughts forgot that you can optionally turn things off.
