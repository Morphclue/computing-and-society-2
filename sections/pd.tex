\section{Participatory Design}
Participatory design is another design method that helps to further increase the acceptance rate of UK4You\cite{participatory-design-spinuzzi}.
It is also often referred to as co-design. 
Participatory design consists of going through the design process together with end users or other stakeholders.
Thus, the wishes of the users are integrated into the platform similarly to the user-centered design methodology.
End users can also draw designers' attention to things that might otherwise not have been noticed.
However, it is important to mention here that with co-design, the main designer remains the decision-maker.\\

In order to implement this design method, two students were asked to become co-designers.
They were shown all the knowledge they had already acquired.
For the final platform design, various snippets were put together for a paper prototype.
Together with the two other students, the individual snippets were rearranged, redesigned or discarded.
After each small iteration, the platform could be improved.
It was always important that the values of productivity and inclusiveness defined at the beginning were observed.\\

As part of the design phase, three elementary website pages were created, which can be found under \autoref{sec:mockups}.
\autoref{fig:timeline} shows the homepage with the timeline as the central element. 
\autoref{fig:company} provides the opportunity to look at the different companies.
Another core feature is the profile page, which is shown in \autoref{fig:profile}.
The designs created are high fidelity mockups created from the paper wireframes which have already gone through several iteration stages.

