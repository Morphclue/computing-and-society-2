\section{Participatory Design}
Participatory design is another design method that helps to further increase the acceptance rate of UK4You\cite{participatory-design-spinuzzi}.
It is also often referred to as co-design. 
Participatory design consists of going through the design process together with end users or other stakeholders.
Thus, the wishes of the users are integrated into the platform similarly to the user-centered design methodology.
End users can also draw designers' attention to things that might otherwise not have been noticed.
However, it is important to mention here that with co-design, the main designer remains the decision-maker.\\

In order to implement this design method, two students were asked to become co-designers.
They were shown all the knowledge they had already acquired.
For the final platform design, various snippets were put together for a paper prototype.
Together with the two other students, the individual snippets were rearranged, redesigned or discarded.
After each small iteration, the platform could be improved.
It was always important that the values of productivity and inclusiveness defined at the beginning were observed.
In particular, it was not developed for users, but for the two personas Stefan and Maria.\\

As part of the design phase, three elementary website pages were created, which can be found under \autoref{sec:mockups}.
\autoref{fig:timeline} shows the homepage with the timeline as the central element. 
\autoref{fig:company} provides the opportunity to look at the different companies.
Another core feature is the profile page, which is shown in \autoref{fig:profile}.
The designs created are high fidelity mockups created from the previously mentioned paper wireframes which have already gone through several iteration stages.\\

First, a central element is described, which can be found in all three mockups.
The navigation bar, which contains the logo, a search bar, and three other buttons.
The first thing you notice is that the navigation bar is designed in the university colors\cite{uni-kassel-colors}.
These are used to further support the university's corporate design.\\

The choices for the logo or the title are not further explained here.
The search bar, on the other hand, needs some explanation because search results need to be filtered on a specific basis.
It is particularly important here that users have the power to decide for themselves how they want their content to be displayed.
For this reason, it should be possible to change the sorting of the search results, the timeline or the company page, similar to Reddit.
Sorting could be ascending or descending based on interactions from other users.
However, other sorting options could also be used, such as relevance to specific users.
For example, in the case of company offers, Stefan could receive job offers with a lot of practical experience and Maria with a lot of scientific progress.
Depending on your preferences, it should also be possible to filter the content according to specific criteria.
So Maria could reduce her posts on the timeline to academic and conferences and Stefan could filter out exactly this content.\\

In addition, there are three different buttons, which indicate some functionalities as an example.
The bell should represent the notification system.
Mentions, answers or very important information for users should be displayed there.
The notification system is also customizable down to the smallest detail.
Finally, the importance of a notification can be weighted differently by students.
It is also important that, except urgent messages from the university, all notifications are opt-in instead of opt-out.
Therefore, users are not spammed at the beginning without having given their consent.

% TODO: navbar-buttons

% profanity word filter -> willi -> also a name
% TODO: Moderation + Report content
% TODO: No ban -> restricted
% TODO: Support in plattform = new jobs

% TODO: suggestions from PD
% TODO: Course to use the software

% Later CS2 presentation:
% TODO: no friends but messaging? -> first idea to avoid showing friend group etc, but too stuck on these thoughts forgot that you can optionally turn things off.
