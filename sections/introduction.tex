\section{Introduction}\label{sec:introduction}
In an era of social distancing and limited contact with others\footnote{Caused by SARS-CoV-2 that reached Germany on 27 January 2020\cite{bundesgesundheitsministerium-corona}}, social media has become an important place for interaction.
Social media offers the opportunity to continue to network, write messages and maintain contacts.
During this time, students at the University of Kassel also had to be informed by email that the university was closed for the time being.
As a result, the network among the students collapsed and an open exchange of knowledge became more complicated.
Some social media such as Facebook or Instagram are operated by the University of Kassel, but this is more in the form of a news channel than an open exchange between students.
Obviously there are still some smaller communication channels like students who have exchanged their number and write on WhatsApp or Telegram, but these groups are very closed and networking is rather slow or non-existent.\\

There is a Discord server at Faculty 16 in particular, which was created on 21 September 2020, to give students the opportunity to meet again in study groups or to ask questions about their studies.
Since then, the Discord server has been a central point of contact and an alternative way for students or university staff to communicate.
Around 800 persons have already joined this social platform.
The problem with this is that only FB16 students are managed on this server.
So there is still a lack of students from other departments.
Even people outside the University of Kassel can very easily join the server with destructive intentions.
People can impersonate others, spam or spread misinformation without much concern.
There are numerous other reasons why a self-hosted platform would make sense.
In the following report, this will be examined in more detail using well-known design methods.\\

These well-known design approaches are called user-centered design, value-sensitive design and participatory design.
The values (e.g. privacy) are placed in the foreground, especially in the value-sensitive design.
In this way, a platform that is as inclusive as possible can be created and possible damage can be minimized.
The productivity of the platform should also be given priority so that students do not have another source of distraction from their studies or life outside social media.
