\section{User-Centered Design}
User-Centered Design (UCD) is a method in which the future users are involved from the start\cite{handbook-usability}.
This ensures that the structure, content and design of the end product are largely driven by the needs, expectations and understanding of the user.
Because a product is only a valuable product if it is actually used by the user.
The closer the product is to the goals and wishes of the users, the more attractive it is for them.
There are different approaches to implement these goals and desires.
Mainly, this can be broken down into three iterative phases.
First, a foundation must be established using various research methods.
Based on this basis, a first conceptual design is developed.
These solutions must then be evaluated together with the users.
This evaluation often leads to new problems, which first have to be investigated, solved, designed and re-evaluated.\\

\subsection{Interviews}
These three phases will also be run through as part of the development of UK4You.
In the first research phase, interviews were carried out to gain different perspectives on the platform to be developed.
A market analysis was also carried out, which will be examined in more detail after this section.
In order to identify as many problems as possible, both bachelor's and master's students were surveyed.
Ideally, employees, doctoral students or professors would also be interviewed at this point.
However, there was not an option to do this during the survey period.
The sample size of $n=5$ is also far too small to be able to make statements about everyone at the University of Kassel.
Nevertheless, better statements can be made about the platform than if the sample size would be $n=1$.\\

The interviews were conducted using an interview guideline (see \autoref{sec:guideline}).
Normally, semi-structured interviews would be conducted at this point because they are more exploratory compared to guided interviews. \textcolor{red}{cite?}
This is because questions are asked in order to shed more light on facts or to explore ideas further.
At this point, however, it was deliberately avoided in order to gain personal experience with structured interviews and to experience the advantages and disadvantages of both interview-types.\\

The interviews were recorded with a microphone after the consent of the individual students.
The audio files were then transcribed into literal, non-phonetic, transcripts.
The audio transcription process offers the advantage that individual passages in a text can be referenced more easily.
Recorded data can also be anonymized and pseudonymised in this way.

\subsubsection{Demographic Data}
The questions in the interview initially related to demographic data.
This data should give more information about preferences based on e.g. the age group or the desired degree.
This data could already contain valuable clues that reveal something about the intended platform design.
Two out of three people who are interested in staying in academia also have a profession that has a university connection. \textcolor{red}{transcripts}
This could suggest that, from the student's perspective, the profession offers an opportunity to enhance success in academia.
With such a small sample size, however, it is unfortunately not really possible to talk about a correlation.
This must also be taken into account for all further statements about the data analysis.\\

Based on the available data, however, it can be seen that the students surveyed want to take different paths in life.
More specifically, there is an academic and professional interest here.
It is important that the platform to be developed offers the opportunity to promote both interests in order to make the platform productive.
Another finding is that all participants are fluent in both German and English.
This can give us more insight into the planned language of the platform.
The University of Kassel requires foreign students to be able to speak the German language.
However, all participants also have the ability to speak English.
This means that nothing would speak against a primarily English platform.
This would also further promote the language skills of the users of the platform.
For foreign students, however, this could be the opposite case, because they do not deepen their knowledge of the German language on the platform.
However, the argument that English is used as the world language and that scientific work is also written in this language is more predominant.\\

\subsubsection{Social media}
After the questions about demographic data, the students were asked about their social media usage behavior.
Criticism of social media was also surveyed.
The first finding was that all users use social media on a daily basis. \textcolor{red}{transcripts}
In a semi-structured interview, you could follow up here to find out what the reasons are or at what times social media is used.
However, as already mentioned, inquiries were deliberately avoided.
This information may indicate that UK4You will also be used on a regular basis.
In order to ensure productivity, care should be taken that users do not stay on the platform for too long and are only presented with relevant information.
Exactly what relevant content is and how it is suggested is explained in more detail later in the report.\\

\begin{figure}[ht]
    \centering
    \begin{tikzpicture}[scale=0.8]
        \pie{12.5/Discord, 12.5/WhatsApp, 18.75/Instagram, 18.75/Facebook, 37.5/Other}
    \end{tikzpicture}
    \caption{Used social media platforms}
    \label{fig:pie-chart-social-media}
\end{figure}

\autoref{fig:pie-chart-social-media} shows the used social media platforms that were identified in the interviews.
It becomes clear that the use of the media is very diverse.
It should be noted here that the students were not strongly guided to recap all social media and may not have listed some.
The definition of social media also seems to be unclear.
The statements \glqq I would argue that Discord is also a social media platform [...]\grqq{} and \glqq [...] WhatsApp if that counts.\grqq{} are strong an indication of this. \textcolor{red}{transcripts}
Various definitions exist, which is why instant messaging services can be partly included as social media and partly excluded.
A paper attempted to summarize these different definitions and four important points were identified\cite{social-media-definition}.
\begin{enumerate}
    \item Social media are Web 2.0 Internet-based applications
    \item User-generated content is the lifeblood of social media
    \item Individuals and groups create user-specific profiles for a site or app designed and maintained by a social media service
    \item Social media services facilitate the development of social networks online by connecting a profile with those of other individuals and/or groups
\end{enumerate}
In this report, instant messaging services are also considered as social media.\\

The Other-category in particular includes two noteworthy platforms.
These two platforms are called Signal and Mastodon\cite{signal}\cite{mastodon}. \textcolor{red}{transcripts}
They are interesting because they are based on values such as privacy and decentralization.
These platforms are also alternatives to existing platforms.
Signal can be seen as a privacy-focused alternative to the popular WhatsApp platform and Mastodon as a decentralized alternative to Twitter.
The other platforms mentioned in the Other-category are ArtStation, YouTube, Twitter and Telegram. \textcolor{red}{transcripts}\\

If you look at the companies behind the mentioned social media, you can see that the company Meta is behind 50\% of the votes.
While only three of the ten platforms used belong to Meta (Facebook, Instagram and WhatsApp), these are also the most used platforms alongside Discord.
This company, like some other social media platforms, has come under criticism.
In 2021, this took on a new dimension when whistleblower Frances Haugen was able to show that Meta was aware of the harmful societal impact of its platforms\cite{whistleblower-meta}.
However, this report is not intended to examine them in detail.
Of course, the points were still taken into account in the later design and are also explained, but no longer explicitly with reference to Meta or other companies.\\

Everyone uses their mobile device to access social media. \textcolor{red}{transcripts}
Three of the five users also use the PC to consume content or write to others. \textcolor{red}{transcripts}
There are various possible reasons why the students surveyed prefer mobile devices.
On the one hand, it could be because it's much more convenient to look at the phone than to boot up the PC.
On the other hand, it could also be due to the fact that the students surveyed also want to consume content outside their own living space.
This can be on a train ride or a walk, for example.
For the UK4You platform design, it can be said that ideally it should be available for both mobile and stationary devices.\\

Another finding is that four out of five users are passive on social media and do not produce their own content.
Unfortunately, it is difficult to draw the conclusion from this finding that mainly passive users can be found on the UK4You platform and only 20\% actively post content.
However, if you add data from other social media, you can see that up to around 80\% of the users of a platform are passive users.
Of course, this varies from platform to platform and could also be completely different for UK4You.
The important lesson to be learned from this is to pay attention to both types of use and not to ignore the passive ones in the design process.
While these may have the least interaction with the platform, they often make up the largest percentage.\\

\subsubsection{Needs for UK4You}

\subsection{Market analysis}
% TODO: Explain market analysis and what the goal is
% TODO: critics on other social media?
% TODO: Market - reflect other design carefully