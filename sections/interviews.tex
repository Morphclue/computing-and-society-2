\section{Interviews}
\textcolor{red}{TODO: rename into UCD?}

As part of the development of UK4You, interviews were initially carried out in order to gain different perspectives on the platform to be developed.
In order to identify as many problems as possible, both bachelor's and master's students were surveyed.
Ideally, employees, doctoral students or professors would also be interviewed at this point.
However, there was not an option to do this during the survey period.
The sample size of $n=5$ is also far too small to be able to make statements about everyone at the University of Kassel.
Nevertheless, better statements can be made about the platform than if the sample size would be $n=1$.\\

The interviews were conducted using an interview guideline (see \autoref{sec:guideline}).
Normally, semi-structured interviews would be conducted at this point because they are more exploratory compared to guided interviews. \textcolor{red}{cite?}
This is because questions are asked in order to shed more light on facts or to explore ideas further.
At this point, however, it was deliberately avoided in order to gain personal experience with structured interviews and to experience the advantages and disadvantages of both interview-types.\\

The interviews were recorded with a microphone after the consent of the individual students.
The audio files were then transcribed into literal, non-phonetic, transcripts.
The audio transcription process offers the advantage that individual passages in a text can be referenced more easily.
Recorded data can also be anonymized and pseudonymised in this way.

The questions in the interview initially related to demographic data.
This data should give more information about preferences based on e.g. the age group or the desired degree.
This data could already contain valuable clues that reveal something about the intended platform design.
Two out of three people who are interested in staying in academia also have a profession that has a university connection. \textcolor{red}{cite from transcripts}
This could suggest that, from the student's perspective, the profession offers an opportunity to enhance success in academia.
With such a small sample size, however, it is unfortunately not really possible to talk about a correlation.
This must also be taken into account for all further statements about the data analysis.\\

Based on the available data, however, it can be seen that the students surveyed want to take different paths in life.
More specifically, there is an academic and professional interest here.
It is important that the platform to be developed offers the opportunity to promote both interests in order to make the platform productive.
Another finding is that all participants are fluent in both German and English.
This can give us more insight into the planned language of the platform.
The University of Kassel requires foreign students to be able to speak the German language.
However, all participants also have the ability to speak English.
This means that nothing would speak against a primarily English platform.
This would also further promote the language skills of the users of the platform.
For foreign students, however, this could be the opposite case, because they do not deepen their knowledge of the German language on the platform.
However, the argument that English is used as the world language and that scientific work is also written in this language is more predominant.\\
