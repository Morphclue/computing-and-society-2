\section{Transcript - Person 5}
\sloppy
\texttt{\begin{itemize}[]
    \setlength\itemsep{0.02em}
    \linenumbers
    \item \interview{IV} How old are you?
    \item \interview{P1} 27.
    \item \interview{IV} What gender do you identify with?
    \item \interview{P1} Male.
    \item \interview{IV} What is the highest degree or level of education you have completed?
    \item \interview{P1} Bachelor.
    \item \interview{IV} Which languages do you speak?
    \item \interview{P1} German and English.
    \item \interview{IV} Is there a connection between your job and the university?
    \item \interview{P1} Yes. My job is in a company that is collaborating with the university on a project.
    \item \interview{IV} Are you interested in staying in academia?
    \item \interview{P1} I could imagine staying. Yes.
    \item \interview{IV} How actively do you use social media platforms?
    \item \interview{P1} Daily.
    \item \interview{IV} Which social media platforms are you using?
    \item \interview{P1} Facebook, Instagram and if you include messenger then also Whatsapp, Telegram and Signal.
    \item \interview{IV} What devices are you using to access social media?
    \item \interview{P1} Mainly on mobile devices, but also on laptops and computers.
    \item \interview{IV} Would you describe yourself as a passive or active social media user?
    \item \interview{P1} Definitely passive.
    \item \interview{IV} Why are you using social media?
    \item \interview{P1} To stay up to date with what acquaintances, relatives or friends are doing. But also for entertainment through videos.
    \item \interview{IV} What do you like about social media?
    \item \interview{P1} The general idea. You are not forced to look through social media for a specific period of time. You can decide on your own how long you want to stay there. You can also access information pretty fast.
    \item \interview{IV} What do you dislike about social media?
    \item \interview{P1} Difficult to answer. That depends entirely on the platform. For example, I don't like it when there's just too much advertising. But wrong information is also a big problem.
    \item \interview{IV} What should be included in your profile?
    \item \interview{P1} So definitely the age. Then maybe the semester in which you can weigh up how much experience the person has already gained in this field. It would also be a good idea to list experiences that were made outside of the university. Then, of course, soft or hard skills, similar to LinkedIn, would also be interesting. So what programming languages or other skills does this person know? Then, of course, roles would also be very interesting. So tutor, student and so on.
    \item \interview{IV} How would you like to share information with other users?
    \item \interview{P1} Similar to Instagram, I think it's actually quite good. But I'm not sure if an endless scroll would be the right idea here. Maybe like Discord where you have individual topics and can send messages there. Maybe then filter options would be a good thing to be able to optimize the lists yourself.
    \item \interview{IV} What would be the main functionalities of such a platform?
    \item \interview{P1} The one just mentioned, private chats and maybe module specific things
    \item \interview{IV} What should happen to users that do no study at University of Kassel anymore?
    \item \interview{P1} I think it would be good if there was a longer period of time before you were removed from the platform.
    \item \interview{IV} Where do you think the strengths of such a platform could be?
    \item \interview{P1} In contrast to other social media, there are really only people from the university here and not all other professional fields. If you compare it to LinkedIn, where people just go looking for jobs UK4You could be a parallel that only specializes on university stuff.
    \item \interview{IV} Where do you think the weaknesses of such a platform could be?
    \item \interview{P1} That it may not be well accepted, because you don't know if people will actively use it because it's something new.
\end{itemize}}
\nolinenumbers
