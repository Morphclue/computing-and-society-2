\section{Transcript - Person 3}
\sloppy
\texttt{\begin{itemize}[]
    \setlength\itemsep{0.02em}
    \linenumbers
    \item \interview{IV} How old are you?
    \item \interview{P3} 23.
    \item \interview{IV} What gender do you identify with?
    \item \interview{P3} Male.
    \item \interview{IV} What is the highest degree or level of education you have completed?
    \item \interview{P3} Abitur.
    \item \interview{IV} Which languages do you speak?
    \item \interview{P3} German, English and Russian.
    \item \interview{IV} Is there a connection between your job and the university?
    \item \interview{P3} No.
    \item \interview{IV} Are you interested in staying in academia?
    \item \interview{P3} Yes, I think.
    \item \interview{IV} How actively do you use social media platforms?
    \item \interview{P3} I use YouTube daily for example. But there are also others that I use regularly.
    \item \interview{IV} Which social media platforms are you using?
    \item \interview{P3} Twitter, Mastodon, Facebook, Discord and YouTube are the most used platforms.
    \item \interview{IV} What devices are you using to access social media?
    \item \interview{P3} Mostly my smartphone, but sometimes my PC.
    \item \interview{IV} Would you describe yourself as a passive or active social media user?
    \item \interview{P3} I would rather describe myself as passive.
    \item \interview{IV} Why are you using social media?
    \item \interview{P3} Just for information and entertainment.
    \item \interview{IV} What do you like about social media?
    \item \interview{P3} The easy accessibility and the possibility to make new contacts.
    \item \interview{IV} What do you dislike about social media?
    \item \interview{P3} That people use them so often that at events or private meetings, everyone just sits on their smartphones.
    \item \interview{IV} What should be included in your profile?
    \item \interview{P3} The profile should definitely indicate whether it is a student or a professor or something. What subject you might be in or what the person is studying. And more general information like age, gender and so on.
    \item \interview{IV} How would you like to share information with other users?
    \item \interview{P3} I think a timeline is the best way to access information and share it with others.
    \item \interview{IV} What would be the main functionalities of such a platform?
    \item \interview{P3} Opportunities to somehow network with other students. It is important to form study groups with the help of UK4You or simply to find people who share the same interests.
    \item \interview{IV} What should happen to users that do no study at University of Kassel anymore?
    \item \interview{P3} That you have an indication on a profile that this person is no longer studying at the university.
    \item \interview{IV} Where do you think the strengths of such a platform could be?
    \item \interview{P3} That a faster personal connection can be established between the students.
    \item \interview{IV} Where do you think the weaknesses of such a platform could be?
    \item \interview{P3} I can imagine that some students think "I already have Instagram, Facebook or something else, why do I need another platform that only has random people from the university". So that they no longer see any new value in it.
\end{itemize}}
\nolinenumbers
