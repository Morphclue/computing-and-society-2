\section{Value Sensitive Design}\label{sec:vsd}
% TODO: Intro VSD - later on reflective design
\textcolor{red}{Intro VSD}
There are many values that are important for the creation of the UK4You platform.
The different values are privacy, trust, ownership, freedom from bias, usability, accessibility and informed consent.
But how are these values actually defined?
Let's first take a look at a well-known dictionary: The Oxford English Dictionary.

% Reason how might the value of privacy be defined?
There are two different definitions from the Oxford English Dictionary\cite{oxford-dictionary}.
The first one defines privacy as the state of being alone and not watched or disturbed by other people.
The second one defines it as the state of being free from the attention of the public.
Let's also look at the etymology, i.e. the word origin, of the word privacy.\\

According to Etymonline, an online etymology dictionary, the word privacy comes from the word private\cite{etymonline}.
It was used around the year 1600 and is the noun of the word private.
Private originally comes from the Latin word privatus, which means "personal or belonging to oneself". % Cite?
This is in contrast to the word public.
The origin comes from a time when things were either owned by the state or by an individual.
For this reason, privatus can also mean "not belonging to the state" and publicus, the Latin translation of public, "belonging to the state". \\ % cite?

The exact historical origins are difficult to define.
It is clear that figures such as Aristotle have already philosophized on this subject\cite{stanford-philosophy}.
He thought about a public sphere, where political activity takes place and a private sphere that is more focused on family and domestic life.
The word "Privatsphäre" (literally privatsphere), which comes from the German and is the translation of privacy, is probably a reference to this idea.
Another definition of the word privacy comes from Old French in the late 14th century and means secret or solitude.\\

The various definitions and ideas were then first described as a human right in 1891\cite{history-of-privacy}.
There, the American lawyers Samuel Warren and Louis Brandeis defined privacy as "the right to be let alone".
In 1967, professor of public law and government Alan Furman Westin published a book called Privacy and freedom\cite{privacy-and-freedom}.
He describes privacy as the claim to determine the extent to which information is shared with others.
To date, however, it is difficult to grasp all the different meanings and interpretations of privacy.
Privacy can therefore be understood as a complicated construct made up of different definitions and contexts in which they take place.
\textcolor{red}{Different definitions? Control over information, critique?}\\

% Is it in conflict with any other values on your list? 
If certain processes or data are not transparent, then this can be due to the affiliation of privacy.
For example, a company does not want to publish certain data sets or processes because others could generate revenue from them.
However, the lack of transparency prevents the building of trust and a subsequent informed consent.
Usability can also be restricted if, for example, a person is not interested in passing on location data.
As a result, further functionalities could be lost and the user experience could deteriorate.

% Can you propose technological solutions that safeguard the value of privacy and make the users aware of privacy as a value? 
\begin{itemize}
    \item Encryption of data
    \item Anonymization
    \item Ownership
    \item Registration with matriculation number = makes impersonating impossible, but still anonymizes the user.
\end{itemize}

% Speculate: should this value be challenged? If so, why and how?
Privacy as a value should be challenged as it is a human right.
There should be the right to switch data back to private.
There are numerous examples that illustrate this.
Taking and posting an unauthorized photo should be reversible.
Likewise, personal data should not be passed on or sold to third parties.
This can have serious consequences in the case of stalking or spamming.
It should always be up to everyone how much they share and publish on the Internet.
